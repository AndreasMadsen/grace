%!TEX root=report.tex
\section{Conclusion}

The main purpose of this report was to describe mass losses or gains at the West Coast of Greenland and Western Antarctica.
Both locations were found to be losing mass.
Both location had periodic trends, but while West Greenland had soft oscillating curves Western Antarctica's mass loss was more jagged this might suggests that ice is melting and breaking of in a more unstable pattern.
However, overall it was found that ice at the two locations was melting with approximately the same speed and with a slight difference in acceleration.

\begin{table}[H]
\centering
\begin{tabular}[H]{l | cc}
Parameter/location & Western Antarctica  & South Coast of Greenland \\ \hline
Velocity $\left[\frac{m}{day}\right]$ &  $-2.26 \cdot 10^{-4}$ & $-2.24 \cdot 10^{-4}$ \\
Acceleration $\left[\frac{m}{day^2}\right]$ &  $-1.24 \cdot 10^{-7}$ & $-8.27 \cdot 10^{-8}$ \\
\end{tabular}
\caption{Velocity and acceleration for the two analysed locations}
\end{table}

Of cause these estimates depends on the selected locations and by choosing some different locations, the estimates may not be approximately the same.
To get a better understanding of the patterns plots of the velocity (Figure \ref{fig:ols-world-parameter-vel}) and acceleration (Figure \ref{fig:ols-world-parameter-acc}) in a world view, was also made.
From these it's seen that Greenland and Antarctica are generally very similar in maximum magnitude of both acceleration and velocity. The most surprising observation to is properly found at Greenland. Here the highest velocity is found on the east coast, but the highest acceleration is actually located on the west cost. This is not similar to Antarctica, where the highest acceleration and velocity is would at approximately the place.

Similar results was gained when using clustering algorithms to find similar locations. Here GMM with a Kernel PCA \ref{section:result-gmm} seems to give a much less noisy result than K-means \ref{section:result-kmeans}.

These results are of cause influenced by the post glacial rebound. Later the ICE-5G dataset was  used to correct for this (glacial isostatic adjustment), though this NASA \cite{nasa-gia-incomplete} suggest that this is not enough to get useful estimates.

Attempts to improving the variance of these estimates was also made. First using by using GLS model instead of a OLS model \ref{section:result-ols-ar} and later by introducing basis expansion \ref{section:result-slines} to allow for more flexibility in the seasons. Both of these gave good results and its even possible to combine the methods. However one should look intro getting a continues result when using splines, otherwise the result will quite theoretical.

Another way to improve the variance is to reduce the amount of OLS parameters. Her the p-values and the LAR \ref{section:result-lar} model suggests that only OLS parameters down to $3 \cdot \omega$ is worth keeping. At least for those selected locations.

Finally a time series model was used in form of a seasonal ARIMA \ref{section:result-ts}. The result here was disappointing in that the residuals is far from being white noise. This suggests that there are exogenous inputs there affects to system in a significant way. Finding and observing those variables would possibly also allow for a better OLS estimate. 
