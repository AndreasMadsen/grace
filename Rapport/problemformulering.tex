%!TEX root=report.tex
\section{Problem definition}
Lately, the climate has become a very hot topic. 
In recent times almost all researchers have reached a consensus; 
The globe is heating up and the primary catalyst of this process are human carbon dioxide emissions.
One of the consequences of a warmer Earth is melting ice at the poles.
This leads to rising oceans which could cause a lot damage to lowlands;
In Denmark an example of the latter would be the marsh area in West Jutland.
Thus, an important question is exactly how the process of ice melting is evolving. 
Is it accelerating? Does it fluctuate? Does ice at the North Pole melt as fast as ice at the South Pole?
 \todo{Tjek at dette er besvaret i konklusionen}

To add some specificity, this report will focus on analysing  local mass losses on the surface of the Earth.
If ice is melting it will show up as a local reduction of mass since liquid water flows and thus will be distributed approximately equally in the ocean. 

Our data source is Gravity and Climate Experiment (GRACE) \cite{GRACE-data-source}. 
GRACE is a government funded research project which commenced in 2002. Data was captured using 2 satellites trailing each other while orbiting the Earth.
 By measuring the distance between the satellites one can estimate the strength of the gravitational field which then can be interpreted as reductions or increases in mass.


This report will seek to uncover locations, which are experiencing a significant mass gain/loss, via mathematical models. 
In addition to analysis of the spatial variance, variation in the time domain will also be analysed - it would be particularly interesting if trends are present.
Finally, results and their uncertainties will be commented on.
