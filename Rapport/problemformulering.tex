%!TEX root=report.tex
\section{Problemformulering}
Klimaet især i det 21. århundrede er blevet et meget debatteret emne. 
I dag er stort set alle forskere enige om at jorden bliver varmere og konsensus er pt. at det primært skyldes menneskelig udledning af drivhusgasser. 
En af konsekvenserne ved en varmere jordklode, er at polerne smelter. 
Dette medfører stigende vandstande, hvilket kan have store konsekvenser for lavlandsområder (i Danmark kunne det f.eks. være marsken i Sønderjylland).
Men hvordan er udviklingen egentlig?
Er den accelererende, hvor meget svinger den, er der forskel på nord og syd. \todo{Er dette besvaret i konklusionen}

Mere præcist vil denne rapport se nærmere på lokale massetab på jordoverfladen. 
Hvis isen smelter vil det give udslag som et massetab, da vandet jo vil blive fordelt i havet.

Vores data kommer fra Gravity and Climate experiment (GRACE) \cite{GRACE-data-source}. 
GRACE er et satellitprojekt der startede i 2002. I projektet indgår 2 satelitter som tilsammen er i stand til at måle på forskelle i jordens tyngdefelt.
Disse forskelle kan fortolkes som lokale massetab eller tilvækster.

Denne rapport vil med en matematisk model søge at afdække, hvilke steder på kloden der oplever signifikante massetab eller massetilvækster.
Udover den geografiske dimension, vil variationen i den tidslige dimensioner også blive undersøgt. 
Specielt interessant er det at kunne identificere eventuelle trends.
Slutteligt vil der blive kommenteret på resultaterne og deres usikkerheder.
