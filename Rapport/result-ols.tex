%!TEX root=report.tex
\subsection{OLS}

The first part of the OLS analysis will look at the EWH at two specific locations. The second part will then focus on the entire world. The main purpose of the OLS analysis is to estimate the velocity and acceleration of the gravity changes.

\subsubsection{Selected locations}

The two selected locations are:
\begin{itemize}
\item Greenland ($63.5^\circ$ N, $49.5^\circ$ W)
\item South Pole ($74.5^\circ$ S, $87.5^\circ$ W)
\end{itemize}
\begin{figure}[H]
	\centering
	\includegraphics[height=6cm]{figures/ols-selected-map}
	\caption{The two selected locations, marked with a red star}
\end{figure}

\paragraph{Greenland}

At the west coast of Greenland is a strong period trend, caused by the the ice melting over the summer and reappearing over the winter. This trend is seen in Figure \ref{fig:ols-selected-0-fit}. In order to estimate the velocity and acceleration of the EWH, there isn't caused by some season, it's important that the periodc trend is caught by the $\sin(\cdot)$ and $\cos(\cdot)$ terms in OLS regression.
\begin{figure}[H]
	\centering
	\includegraphics[width=\textwidth]{figures/ols-selected-0-fit}
	\caption{Messurements are blue, the OLS fit is red.}
	\label{fig:ols-selected-0-fit}
\end{figure}
 
From Figure \ref{fig:ols-selected-0-fit} it's seen that the period trend is caught by the model, however for some seasons (particular between 2008 and 2010) the fit is not very good. This is even more clear when looking at the residuals (Figure \ref{fig:ols-selected-0-residual}). From this it's clear that the residuals are far from white noise, which was one of the OLS assumptions.
\begin{figure}[H]
	\centering
	\includegraphics[width=\textwidth]{figures/ols-selected-0-residual}
	\caption{The OLS residuals are blue.}
	\label{fig:ols-selected-0-residual}
\end{figure}

Besides the yearly season, there is when looking at the estimated OLS coefficients (Table \ref{table:ols-selected-0-paramters}) also a half yearly season. Though strangely enough it isn't very visible on the OLS fit (Figure \ref{fig:ols-selected-0-fit}). The remaining seasons are with 95\% confidence not significantly different from zero and for the purpose of getting a better estimate, it might be worth iteratively removing these terms.
\begin{table}[H]
\centering
\centerline{\begin{tabular}{r|rr}
\hline
 name                           & estimate               & p-value   \\
\hline
 $intercept (1)$                & $-1.77 \cdot 10^{-01}$ & $0.000$   \\
 $vel. (t)$                     & $-2.24 \cdot 10^{-04}$ & $0.000$   \\
 $acc. (0.5 \cdot t^2)$         & $-8.27 \cdot 10^{-08}$ & $0.000$   \\
 $\cos(1 \cdot \omega \cdot t)$ & $-1.35 \cdot 10^{-02}$ & $0.000$   \\
 $\sin(1 \cdot \omega \cdot t)$ & $-8.28 \cdot 10^{-02}$ & $0.000$   \\
 $\cos(2 \cdot \omega \cdot t)$ & $-5.86 \cdot 10^{-03}$ & $0.045$   \\
 $\sin(2 \cdot \omega \cdot t)$ & $-1.41 \cdot 10^{-02}$ & $0.000$   \\
 $\cos(3 \cdot \omega \cdot t)$ & $-5.01 \cdot 10^{-03}$ & $0.089$   \\
 $\sin(3 \cdot \omega \cdot t)$ & $-5.69 \cdot 10^{-03}$ & $0.053$   \\
 $\cos(4 \cdot \omega \cdot t)$ & $-5.49 \cdot 10^{-03}$ & $0.060$   \\
 $\sin(4 \cdot \omega \cdot t)$ & $-1.66 \cdot 10^{-03}$ & $0.573$   \\
 $\cos(5 \cdot \omega \cdot t)$ & $-2.19 \cdot 10^{-03}$ & $0.455$   \\
 $\sin(5 \cdot \omega \cdot t)$ & $-1.93 \cdot 10^{-03}$ & $0.511$   \\
 $\cos(6 \cdot \omega \cdot t)$ & $4.29 \cdot 10^{-04}$  & $0.883$   \\
 $\sin(6 \cdot \omega \cdot t)$ & $1.43 \cdot 10^{-03}$  & $0.627$   \\
 $\cos(7 \cdot \omega \cdot t)$ & $9.55 \cdot 10^{-04}$  & $0.744$   \\
 $\sin(7 \cdot \omega \cdot t)$ & $1.09 \cdot 10^{-04}$  & $0.970$   \\
 $\cos(8 \cdot \omega \cdot t)$ & $-3.08 \cdot 10^{-04}$ & $0.916$   \\
 $\sin(8 \cdot \omega \cdot t)$ & $-2.37 \cdot 10^{-04}$ & $0.935$   \\
 $\cos(9 \cdot \omega \cdot t)$ & $-2.02 \cdot 10^{-03}$ & $0.491$   \\
\hline
\end{tabular}\hspace{1cm}\begin{tabular}{r|rr}
\hline
 name                            & estimate               & p-value   \\
\hline
 $\sin(9 \cdot \omega \cdot t)$  & $8.48 \cdot 10^{-04}$  & $0.772$   \\
 $\cos(10 \cdot \omega \cdot t)$ & $1.54 \cdot 10^{-03}$  & $0.599$   \\
 $\sin(10 \cdot \omega \cdot t)$ & $-5.77 \cdot 10^{-04}$ & $0.843$   \\
 $\cos(11 \cdot \omega \cdot t)$ & $-5.95 \cdot 10^{-04}$ & $0.839$   \\
 $\sin(11 \cdot \omega \cdot t)$ & $-1.47 \cdot 10^{-03}$ & $0.615$   \\
 $\cos(12 \cdot \omega \cdot t)$ & $1.78 \cdot 10^{-04}$  & $0.952$   \\
 $\sin(12 \cdot \omega \cdot t)$ & $5.37 \cdot 10^{-04}$  & $0.854$   \\
 $\cos(13 \cdot \omega \cdot t)$ & $2.57 \cdot 10^{-03}$  & $0.383$   \\
 $\sin(13 \cdot \omega \cdot t)$ & $-6.35 \cdot 10^{-04}$ & $0.827$   \\
 $\cos(14 \cdot \omega \cdot t)$ & $3.16 \cdot 10^{-04}$  & $0.914$   \\
 $\sin(14 \cdot \omega \cdot t)$ & $-2.99 \cdot 10^{-04}$ & $0.919$   \\
 $\cos(15 \cdot \omega \cdot t)$ & $-7.23 \cdot 10^{-04}$ & $0.804$   \\
 $\sin(15 \cdot \omega \cdot t)$ & $3.33 \cdot 10^{-04}$  & $0.910$   \\
 $\cos(16 \cdot \omega \cdot t)$ & $-8.07 \cdot 10^{-04}$ & $0.785$   \\
 $\sin(16 \cdot \omega \cdot t)$ & $-6.93 \cdot 10^{-04}$ & $0.811$   \\
 $\cos(17 \cdot \omega \cdot t)$ & $-1.26 \cdot 10^{-03}$ & $0.665$   \\
 $\sin(17 \cdot \omega \cdot t)$ & $-7.28 \cdot 10^{-04}$ & $0.803$   \\
 $\cos(18 \cdot \omega \cdot t)$ & $-4.93 \cdot 10^{-04}$ & $0.864$   \\
 $\sin(18 \cdot \omega \cdot t)$ & $-9.94 \cdot 10^{-06}$ & $0.997$   \\
                                 &                        &           \\
\hline
\end{tabular}}
\caption{Parameter esimates $\hat{\beta}$ and their p-values. }
\label{table:ols-selected-0-paramters}
\end{table}

\paragraph{South Pole}

Just like with Greenland there is a clear drop in mass as seen in Figure \ref{fig:ols-selected-1-fit}. However interestingly enough there is no seasonal trend or at least it's not very strong. Also the existence of high frequency terms on the OLS regression, is more apparent. Though from the actual data it don't look like such period trend exists. This suggests that to many frequency terms have been used, thus causing some overfitting.
\begin{figure}[H]
	\centering
	\includegraphics[width=\textwidth]{figures/ols-selected-1-fit}
	\caption{Messurements are blue, the OLS fit is red.}
	\label{fig:ols-selected-1-fit}
\end{figure}

When looking at the residuals in Figure \ref{fig:ols-selected-1-residual}, it's again clear that the residuals are far from being white noise. There also seams to be some periodic trend in the residuals, though the beginning and the end of these seasons is quite hard to make out. 

\begin{figure}[H]
	\centering
	\includegraphics[width=\textwidth]{figures/ols-selected-1-residual}
	\caption{The OLS residuals are blue.}
	\label{fig:ols-selected-1-residual}
\end{figure}

Just as with Greenland many of the OLS terms are with 95\% confidence not significantly different from zero. Interestingly enough there is a yearly season when looking at Table \ref{table:ols-selected-1-paramters}, though it's very hard to make out from the OLS fit in Figure \ref{fig:ols-selected-1-fit}.

\begin{table}[H]
\centering
\centerline{\begin{tabular}{r|rr}
\hline
 name                           & estimate               & p-value   \\
\hline
 $intercept (1)$                & $-1.38 \cdot 10^{-01}$ & $0.000$   \\
 $vel. (t)$                     & $-2.26 \cdot 10^{-04}$ & $0.000$   \\
 $acc. (0.5 \cdot t^2)$         & $-1.24 \cdot 10^{-07}$ & $0.000$   \\
 $\cos(1 \cdot \omega \cdot t)$ & $1.69 \cdot 10^{-02}$  & $0.000$   \\
 $\sin(1 \cdot \omega \cdot t)$ & $1.42 \cdot 10^{-02}$  & $0.000$   \\
 $\cos(2 \cdot \omega \cdot t)$ & $-7.19 \cdot 10^{-03}$ & $0.046$   \\
 $\sin(2 \cdot \omega \cdot t)$ & $2.82 \cdot 10^{-03}$  & $0.440$   \\
 $\cos(3 \cdot \omega \cdot t)$ & $-4.36 \cdot 10^{-03}$ & $0.227$   \\
 $\sin(3 \cdot \omega \cdot t)$ & $-1.23 \cdot 10^{-03}$ & $0.732$   \\
 $\cos(4 \cdot \omega \cdot t)$ & $1.78 \cdot 10^{-03}$  & $0.619$   \\
 $\sin(4 \cdot \omega \cdot t)$ & $2.81 \cdot 10^{-03}$  & $0.437$   \\
 $\cos(5 \cdot \omega \cdot t)$ & $2.45 \cdot 10^{-03}$  & $0.496$   \\
 $\sin(5 \cdot \omega \cdot t)$ & $2.53 \cdot 10^{-03}$  & $0.484$   \\
 $\cos(6 \cdot \omega \cdot t)$ & $-1.60 \cdot 10^{-03}$ & $0.654$   \\
 $\sin(6 \cdot \omega \cdot t)$ & $-8.66 \cdot 10^{-04}$ & $0.811$   \\
 $\cos(7 \cdot \omega \cdot t)$ & $8.95 \cdot 10^{-04}$  & $0.803$   \\
 $\sin(7 \cdot \omega \cdot t)$ & $-3.27 \cdot 10^{-03}$ & $0.364$   \\
 $\cos(8 \cdot \omega \cdot t)$ & $1.62 \cdot 10^{-03}$  & $0.652$   \\
 $\sin(8 \cdot \omega \cdot t)$ & $-1.51 \cdot 10^{-03}$ & $0.674$   \\
 $\cos(9 \cdot \omega \cdot t)$ & $3.76 \cdot 10^{-04}$  & $0.917$   \\
\hline
\end{tabular}\hspace{1cm}\begin{tabular}{r|rr}
\hline
 name                            & estimate               & p-value   \\
\hline
 $\sin(9 \cdot \omega \cdot t)$  & $2.05 \cdot 10^{-03}$  & $0.568$   \\
 $\cos(10 \cdot \omega \cdot t)$ & $-4.54 \cdot 10^{-04}$ & $0.900$   \\
 $\sin(10 \cdot \omega \cdot t)$ & $-1.26 \cdot 10^{-03}$ & $0.726$   \\
 $\cos(11 \cdot \omega \cdot t)$ & $-1.21 \cdot 10^{-04}$ & $0.973$   \\
 $\sin(11 \cdot \omega \cdot t)$ & $-1.07 \cdot 10^{-03}$ & $0.766$   \\
 $\cos(12 \cdot \omega \cdot t)$ & $-1.20 \cdot 10^{-03}$ & $0.740$   \\
 $\sin(12 \cdot \omega \cdot t)$ & $-8.20 \cdot 10^{-04}$ & $0.819$   \\
 $\cos(13 \cdot \omega \cdot t)$ & $-5.73 \cdot 10^{-03}$ & $0.114$   \\
 $\sin(13 \cdot \omega \cdot t)$ & $-8.50 \cdot 10^{-04}$ & $0.812$   \\
 $\cos(14 \cdot \omega \cdot t)$ & $-6.55 \cdot 10^{-04}$ & $0.855$   \\
 $\sin(14 \cdot \omega \cdot t)$ & $2.25 \cdot 10^{-03}$  & $0.533$   \\
 $\cos(15 \cdot \omega \cdot t)$ & $-4.32 \cdot 10^{-04}$ & $0.904$   \\
 $\sin(15 \cdot \omega \cdot t)$ & $3.05 \cdot 10^{-04}$  & $0.933$   \\
 $\cos(16 \cdot \omega \cdot t)$ & $3.51 \cdot 10^{-04}$  & $0.923$   \\
 $\sin(16 \cdot \omega \cdot t)$ & $-3.73 \cdot 10^{-04}$ & $0.917$   \\
 $\cos(17 \cdot \omega \cdot t)$ & $1.22 \cdot 10^{-03}$  & $0.735$   \\
 $\sin(17 \cdot \omega \cdot t)$ & $-1.11 \cdot 10^{-03}$ & $0.758$   \\
 $\cos(18 \cdot \omega \cdot t)$ & $-1.92 \cdot 10^{-04}$ & $0.957$   \\
 $\sin(18 \cdot \omega \cdot t)$ & $7.18 \cdot 10^{-04}$  & $0.844$   \\
                                 &                        &           \\
\hline
\end{tabular}}
\caption{Parameter esimates $\hat{\beta}$ and their p-values. }
\label{table:ols-selected-1-paramters}
\end{table}

\pagebreak
\subsubsection{World view}

\paragraph{Parameters}

For interpreting climatic changes there can be seen as mass changes, the velocity and acceleration of these mass changes are the most important parameters in the OLS regression.
\begin{figure}[H]
	\centering
	\includegraphics[width=\textwidth]{figures/ols-world-parameter-vel}
	\caption{Estimated velocity ($t$) parameters}
	\label{fig:ols-world-parameter-vel}
\end{figure}

\begin{figure}[H]
	\centering
	\includegraphics[width=\textwidth]{figures/ols-world-parameter-acc}
	\caption{Estimated acceleration ($\frac{1}{2} t^2$) parameters}
	\label{fig:ols-world-parameter-acc}
\end{figure}

The first two figures (Figure \ref{fig:ols-world-parameter-vel} and Figure \ref{fig:ols-world-parameter-acc}) shows that there generally is a mass loss at both Greenland and the South Pole and that it is accelerating. However interestingly enough the east cost of Greenland don't show any acceleration, maybe even a slight deceleration of the mass loss (a positive value).

On both figures there is also a circular pattern at Thailand (10 N, 95 E), which is caused by an earthquake in XXXX \todo{insert date}, resulting in significant changes in the mass distribution, thus affecting the data from GRACE.
\begin{figure}[H]
	\centering
	\includegraphics[width=\textwidth]{figures/ols-world-parameter-year}
	\caption{Estimated phase and amplitude for the yearly seasons.}
	\label{fig:ols-world-parameter-year}
\end{figure}

The seasonal trends don't reveal any information about mass loss dude to ice melting, since this is something there will continue to have effect for many years. However it dose affect the estimation of the velocity and acceleration. Also having more season related terms in the OLS regression than necessary, will result fewer degrees of freedom.

In Figure \ref{fig:ols-world-parameter-year} its seen from the intensity, that South Pole don't have a strong yearly seasonal trend unlike Greenland. It also appears that it's primarily the east coast of Greenland there have a yearly seasonal trend. The strongest seasonal trend appears in South America, this is dude to the rainy season. This rain also takes quite some time to reach the ocean, thus causing the phase gradient.

\paragraph{Performance} 

As a measure of performance the root mean squared error have been calculated for each position.
\begin{figure}[H]
	\centering
	\includegraphics[width=\textwidth]{figures/ols-world-performance-rmse}
	\caption{RMSE for each position.}
	\label{fig:ols-world-performance-rmse}
\end{figure}

From \ref{fig:ols-world-performance-rmse} its seen that the hardest places to fit are in South America and Thailand. In South America it is especially around the Amazon River the RMSE is high, this suggests that it is caused by the rainy seasons, since this where also the place with the highest amplitude on the yearly seasonal term. It's a bit odd that this is so hard to fit, since the model do support seasonal trends. One reason could be that this OLS model assumes that the season are equally long each year, which is not very likely. The circular pattern around Thailand is no surprise, since the OLS model was never meant to fit this distortion.

Neither of these issues are related to the ice melting, wish makes them less important. However the east coast of Greenland and the South Pole also show a relative higher RMSE, when comparing to the ocean or the nearby land, which is less ideal. The fact that it is primarily the east coast of Greenland and not the west coast, suggests that the issues are related to the seasons, just like with South America. This was also something there where observed, when looking at the specific locations.

\paragraph{Standard diagnostics}
