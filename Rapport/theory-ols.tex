%!TEX root=report.tex
\subsection{OLS}
OLS (Ordinary Least Squares) regression is used for finding the best linear transformation of one or more exogenous variables $X$ so as to predict the dependent variable $Y$. \todo{Improve formulation}
This is done by a matrix-vector product between a $\beta$ vector and $X$.
Since $X$ can be custom tailored for various purposes it is often referred to as the design matrix.
\begin{align}
Y=X \beta +\epsilon && \text{, where } \mathrm{E}[\epsilon] = 0 \text{ and } \mathrm{D}[\epsilon] = \mathrm{D}[Y] = \sigma^2 I .
\end{align}

In the equation above the $\beta$-vector is the unknown parameter which needs to be determined while  $\epsilon$ is a vector containing the residuals which the model cannot account for.
Often the purpose of OLS is to predict $Y$, but in this case we want to analyze the individual $\beta_i$ elements.

\subsubsection{The solution to the OLS problem}
As the name sugest, the OLS method minimizes the sum of squared residuals ($\epsilon^T \epsilon$).
It is seen directly that $\epsilon = Y - X \beta$ and thus the following is obtained
\begin{equation}
\begin{split}
\epsilon^T\epsilon&=(Y-X\beta)^T (Y-X\beta)\\
&=(Y^T-\beta^T X^T) (Y-X\beta) \\
&=Y^T Y-\beta^T X^T Y-Y^T X \beta + \beta^T X^T X \beta \\
&=Y^T Y- 2\beta^T X^T Y+ \beta^T X^T X \beta.
\end{split}
\end{equation}

One can now differentiate with respect to the $\beta$ vector
\begin{equation}
\begin{split}
\frac{\partial \epsilon^T\epsilon}{\partial \beta}&=-2 X^T Y+2X^T X \beta=2(-X^T Y+X^T X \beta).
\end{split}
\end{equation}

Now  $\epsilon^T \epsilon$'s minimum is given by solving for $\frac{\partial \epsilon^T\epsilon}{\partial \beta} = 0$
\begin{equation}
\begin{split}
\frac{\partial \epsilon^T\epsilon}{\partial \beta} = 0 \Rightarrow 2(-X^T Y+X^T X \hat{\beta}) &= 0 \\
X^T X \hat{\beta}&=X^T Y \\
\hat{\beta}&=(X^T X)^{-1} X^T Y.
\end{split}
\end{equation}

The solution above is formally correct \cite[p.~12]{statistical-learning}, but if $X$ is badly conditioned $(X^T X)^{-1}$ might not be numerically stable (i.e. if the columns in $X$ are highly correlated leading to a near singular $X^T X$) \cite[p.~8]{aasbjerg-ls}.
To avoid this problem $X$ should be factorised and the solution reformulated; in this report we will exclusively do this using SVD
\begin{equation}
\begin{split}
\left( U \Sigma V^T\right)^T \left(U \Sigma V^T\right) \hat{\beta} &= \left(U \Sigma V^T\right)^T Y \\
\left( V \Sigma^2 V^T \right) \hat{\beta} &= V \Sigma U^T Y \\
\hat{\beta} &= \left( V \Sigma^2 V^T \right)^{-1} V \Sigma U^T Y \\
\hat{\beta} &= V \Sigma^{-1} U^T Y.
\end{split}
\end{equation}

It should be noted, that when multiple $\hat{\beta}$ vectors need to be calculated (one vector for each spatial location on the surface) optimization is possible, by arranging $Y$ as a matrix with each column corresponding to a location.
$\hat{\beta}$ will then be a matrix containing all solutions instead of a vector containing only one solution.

\subsubsection{The ``Hat" matrix}
In the special case of the GRACE data, the $X$ matrix is identical for every position. This can be exploited by constructing a hat matrix $H$ which only depends on $X$ and projects $Y$ onto $\hat{Y}$ (puts the hat on $Y$).
\begin{equation}
\begin{split}
\hat{Y} &= X \hat{\beta} \Rightarrow \hat{Y} = X V \Sigma^{-1} U^T Y \\
\hat{Y} &= H X \quad \text{, where } H = X V \Sigma^{-1} U^T.
\end{split}
\end{equation}

As earlier with $\beta$, the hat matrix can be calculated for all $Y$ vectors by horizontally stacking $Y$ to form a matrix.

An important property of the $H$ matrix is that it is idempotent ($H^2 = H$).
This is because $H$ is a projection matrix which projects $Y$ onto $\hat{Y}$.
Projecting $Y$ onto $\hat{Y}$ and then projecting again onto $\hat{Y}$, will obviously not change anything, because one is already in the $\hat{Y}$-plane.

\subsubsection{Root Mean Squared Error}

``Root Mean Squared Residuals'' (RMSE) is an indicator of how good an $Y$ estimate is.
It can be calculated as
\begin{equation}
\hat{\sigma} = \sqrt{\frac{\left(Y - \hat{Y}\right)^T \left(Y - \hat{Y}\right)}{N-p}},
\end{equation}

where $p$ is the number of parameters (elements in $\beta$) and $N$ is the number of observations (elements in $Y$).
RMSE is an estimate for the standard deviation of $Y$ \cite[theorem~3.4]{time-series-analysis} thus the symbol $\hat{\sigma}$.


\subsubsection{The variance of $\hat{Y}$}

The dispersion (variance-covariance) of $\hat{Y}$ can be calculated as
\begin{equation}
\mathrm{D}[\hat{Y}] = \mathrm{D}[H Y] = H \mathrm{D}[Y] H^T = \sigma^2 H^2 = \sigma^2 H.
\end{equation}

The variance of $\hat{Y_i}$ is given by the diagonal elements in $\mathrm{D}[\hat{Y}]$
\begin{equation}
\mathrm{Var}[\hat{Y_i}] = \sigma^2 H_{ii}.
\end{equation}

Because $\sigma^2$ is a scalar, the diagonal in $H$ is important to examine, since it can reveal potential elements (in our case points of time) with high variance in the predictions.

\subsubsection{The dispersion of $\hat{\beta}$}

The dispersion of $\hat{\beta}$ is calculated as \cite[theorem~3.2]{time-series-analysis}:
\begin{equation}
\mathrm{D}[\hat{\beta}] = \sigma^2 (X^T X)^{-1} = \sigma^2 V \Sigma^{-2} V^T
\end{equation}

Seeing as $\sigma^2$ is a scalar and dependent of $Y$, a similar spatial independent expression can be made, by removing the scalar $\sigma^2$  factor, thus looking exclusively at $V \Sigma^{-2} V^T$.

\subsubsection{p-values for $\hat{\beta}$}

To calculate p-values for OLS parameters the student's t-distribution is used, with the t-score \cite[p.~172]{time-series-analysis}
\begin{align}
\mathrm{t} = \frac{\hat{\beta_i}}{\mathrm{SD}[\hat{\beta_i}]} && \text{where: } \mathrm{SD}[\hat{\beta_i}] = \sqrt{\mathrm{Cov}[\hat{\beta}]_{ii}} = \hat{\sigma} \sqrt{ (V \Sigma^{-2} V^T)_{ii} }.
\end{align}

Now by plugging the t-score into the Student's Cumulative T-Distribution Function ($\Phi_t$) with $N - p$ degrees of freedom, we get
\begin{equation}
p = 2 \cdot \Phi_t\left(\mathrm{abs}(t), N-p\right)
\end{equation}

The null-hypothesis is that $\beta_i = 0$ and th  e alternate hypothesis is $\beta_i \not = 0$.
