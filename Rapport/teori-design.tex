%!TEX root=report.tex
\subsection{Design matricen $X$}
Design matricen er $X$ matrix i OLS regression, hvis kolonner indgå i en linear kombination til at beskrive $Y$.
I tilfældet for GRACE data er de eneste ``$x$'' værdier tiden ($\mathbf{t}$), og som udgangspunkt vil $X$ derfor have formatet:
\begin{equation}
X = \left[\begin{matrix} \mathbf{1} & \mathbf{t} \end{matrix}\right].
\end{equation} 
Dette vil betyde at $\hat{Y}$ skal beskrives som $\beta_1 + \beta_2 t$, hvor $\beta_1$ vil være startværdien (uinteressant) og $\beta_2$ vil være hastigheden i udviklingen.
Det fremkommer således naturligt at også have accelerationen i udviklingen med:
\begin{equation}
X = \left[\begin{matrix} \mathbf{1} & \mathbf{t} & \frac{1}{2} \mathbf{t}^2 \end{matrix}\right]
\end{equation} 

Fra Fourier analyse vides det, at funktioner kan anproximeres en uendelige sum af sinus og cosinus funktioner.
Disse trigonometriske funktioner er periodiske og kan således indeholde periodiske signaler, der måtte forkomme i den funktion, man ønsker at approximere.

I GRACE datasættet er der dog ikke en uendelig lille afstand (i tiden) imellem målingerne, en uendelig række af sinus og cosinus funktioner benyttes derfor ikke.
I stedet antages det, at den største forekommende svingning er en årssvingning ($365.242$ dage).
Den mindste svingning(frekvens) som man kan observere når man sampler et signal er Nyquist frekvensen.
I GRACE tilfældet er der 10 dage imellem hver måling, Nyquist frekvensen bliver så $\frac{1}{2 \cdot 10} = \sfrac{1}{20}$ dage.
Dette svarer ca til 18 svingninger pr. år da $\sfrac{365.242}{18} \approx 20.29$. Således fås den endelige design matrix
\begin{equation*}
\resizebox{\textwidth}{!}{$
X = \left[\begin{matrix}
	\mathbf{1} &
	\mathbf{t} &
	\frac{1}{2} \mathbf{t}^2 &
	\cos\left( \frac{2 \pi}{\frac{365.242}{1}} \mathbf{t} \right) &
	\sin\left( \frac{2 \pi}{\frac{365.242}{1}} \mathbf{t} \right) &
	\cdots &
	\cos\left( \frac{2 \pi}{\frac{365.242}{18}} \mathbf{t} \right) &
	\sin\left( \frac{2 \pi}{\frac{365.242}{18}} \mathbf{t} \right)
\end{matrix}\right].
$}
\end{equation*}

