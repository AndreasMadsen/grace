%!TEX root=report.tex
\section{Introduction}
Lately, the climate has become a very hot topic. 
In recent times most natural science researchers have reached a consensus;
the globe is heating up and the primary catalyst of this process is human carbon dioxide emission.
One of the consequences of a warmer Earth is melting ice at various locations.
This leads to rising oceans which could cause a lot of damage to lowlands.
In Denmark an example of the latter would be the marsh area in West Jutland.
Thus, an important question is exactly how the process of ice melting is evolving.

Our data source is Gravity and Climate Experiment (GRACE) \cite{GRACE-data-source}. 
GRACE is a government funded research project which began collecting data in 2002, its mission is to track the gravitational changes on Earth.
The data is captured using two satellites trailing each other while orbiting the Earth.
By measuring the distance between the satellites one can estimate the strength of the gravitational field.
Since gravity is caused by mass, the data can be interpreted as reductions or increases in mass, which may be caused by ice melting.

This report will seek to uncover locations, which are experiencing a significant mass gain/loss, using mathematical models. 
In addition to analysis of the spatial variance, variation in the time domain will also be analyzed.
Finally, results and their uncertainties will be commented on.

\section{Problem definition}
Specifically this report will focus on analyzing local mass losses on the surface of the Earth, partially those changes near Greenland and the Antartica.

\begin{itemize}
\item Is the changes accelerating or decelerating?
\item Does it fluctuate?
\item Does ice at the coasts of the Antartica melt as fast as ice at the coasts of Greenland?
\end{itemize}
