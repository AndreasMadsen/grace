%!TEX root=report.tex
\subsection{Reservations: Glacial Isostatic Adjustment}

The GRACE data that has been used in this report has not been corrected for Glacial Isostatic Adjustment (GIA) (sometimes also referred to as Post-glacial rebound) . 
GIA is an effect that makes land either rise, fall or shift horizontally. 
It can be observed at all locations which during the last ice age was covered by a thick layer of ice. 
The sheer weight of the ice compressed the crust of the Earth so much, that when the ice retracted and the downward force was removed the crust started to move.
 It is estimated that these movements will take many thousand years to subside. \\
Unfortunately the GRACE data does not adjust for the GIA effect.
 Thus it is impossible to know whether observed EWH patterns are actually mass losses/gains or just noise from GIA.
 Region specific averaging kernels are needed to properly account for GIA as well as noise from nearby land hydrology.
According to NASA \todo{insert source} the GRACE data is not suited for Cryospheric studies (ice mass changes) and thus one should keep these reservations in mind when viewing the results.
