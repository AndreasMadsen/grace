%!TEX root=report.tex
\subsection{Time Series Analysis}

In time series analysis one attempts to find a model for predicting future observations
(this is different form the OLS analysis which tries to estimate the trends in the data, such as velocity and acceleration).
 Once such a model is found, it should be validated by analyzing the residuals. These residuals must be approximately white noise distributed \cite[p.~130]{time-series-analysis}, otherwise the model is invalid. 

\subsubsection{ARIMA}

ARIMA is the name of the stochastic model used. The theory is very complex so it will not be discussed in detail here, instead we refer to \cite[p.~130]{time-series-analysis}.

In short ARIMA uses the previous values with some weight to predict future values. Which values are used is denoted by the following notation:
\begin{equation}
(p, d, q) \times (P, D, Q)_s
\end{equation}

$p$ is the highest lags of actual measurements used and $q$ is the maximum amount of lags in residuals used. The $d$ part indicates how many difference operators which should be used to transform the data. $(P, D, Q)$ is completely similar, but steps not by one lag but by $s$ lags ($s=$season length). This allows for seasons trends such as an yearly pattern.

\subsubsection{AutoCorrelation function (ACF) and Partial AutoCorrelation Function (PACF)}
 
ACF and PACF are measures of correlation between different time lags and are used to make a qualified guess about how the prediction model should look like. 
For more detail on how these are estimatied, please see \cite[p.~146]{time-series-analysis}.

When ACF and PACF have been estimated, rules \cite[table~6.1]{time-series-analysis} for how the stochastic model should look can be applied.
Typically when dealing with complex models, this becomes an iterative and subjective process - different analysts might reach different models.
The iterative step consist of estemating parameters, then calculate ACF and PACF, interpet them, estimate parameters, ... , and so it continues until the ACF and PACF suggest that no more lag terms(weights) are needed.

\subsection{Ljung-Box test}

The null hypothesis for the Ljung-Box is ``The data is independently distributed''.
In the ARIMA case it is applied to model residuals; so in other words it tests whether residuals at different lags have any correlation with each other. 
 Thus a low p-value means that the residuals aren't white noise.

The Ljung-Box is a $\chi^2$ test with the statistical value:
\begin{equation}
L = n \cdot (n + 2) \sum_{k=1}^h \frac{\hat{\rho}_k^2}{n - k}
\end{equation}

$n$ is the number of observations. The parameter $h$ is the highest lag in the ACF which should be considered. The $\chi^2$ statistics, have $h - (p + q + P + Q)$ degrees of freedom.
