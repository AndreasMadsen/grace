%!TEX root=report.tex
\subsection{The Gaussian Mixture Model (GMM)}

The GMM is a more advanced clustering model than KMeans. 
The main advantage is that there are no restrictions on each cluster's covariance matrix where as the reader remembers that KMeans clusters had a diagonal covariance structure. 
Such restrictions (i.e. shared covariance, diagonal covariance, spherical covariance etc.) can relatively easily be applied in the GMM if one so chooses; here, however, only the case with no covariance restrictions will be described.
\\
In the GMM each cluster is viewed as a Gaussian density function. 
This density has a centroid (Just as with KMeans) and a covariance matrix.
The density function is assumed to be a combination (mixture) of $K$ Gaussian PDFs (Probability Density Function) where K is finite.


In practice if K is large and the vector space has is high dimensional 

\subsubsection{Dimensionality reduction}