%!TEX root=report.tex
\subsection{Basis expansion}

Basis expansion is done by adding more columns to design matrix. Let us first motivate why this might be needed.

\subsection{Motivation}
It has previously been described how the design matrix $X$ was constructed such that we had 
\begin{equation*}
\resizebox{\textwidth}{!}{$
X = \left[\begin{matrix}
	\mathbf{1} &
	\mathbf{t} &
	\frac{1}{2} \mathbf{t}^2 &
	\cos\left( \dfrac{2 \pi}{\frac{365.242}{1}} \mathbf{t} \right) &
	\sin\left( \dfrac{2 \pi}{\frac{365.242}{1}} \mathbf{t} \right) &
	\cdots &
	\cos\left( \dfrac{2 \pi}{\frac{365.242}{18}} \mathbf{t} \right) &
	\sin\left( \dfrac{2 \pi}{\frac{365.242}{18}} \mathbf{t} \right)
\end{matrix}\right].
$}
\end{equation*}
The problem with the above $X$ is that the periodic columns assumes the pattern accounts for the entire period. This causes problems in areas such as Denmark, where the winter comes later than ``usual'', or the summer is much colder/warmer etc..

This leads to higher variance of the residuals which decreases the accuracy of the velocity and acceleration parameter of the model. By adding columns to the design matrix, such that the periodic pattern is only assumed extend over one period, the accuracy can be improved.

\subsection{The Hinge-function}

One can achieve this basis expansion is by virtue of the hinge-function
\begin{equation}
f(x - \zeta)_+ = \begin{cases}
  f(x - \zeta) & \text{if } x - \zeta \ge 0 \\
  0                 & \text{otherwise}
\end{cases} \quad \forall x \in \mathbb{R}, \zeta \in \mathbb{R}
\end{equation}
$\zeta$ denotes the ``knot'' between the zero part and the $f$ part. By choosing $f$ and adding the hinge function to the design matrix, a basis expansion where terms are only active for some parts of the time series is obtained.

When $f$ is a polynomial one achieve a piecewise polynomial which is called a spline.
In this case each hinge function is guaranteed to be continuous and it can be shown that the spline's derivatives are continuous to order $M-2$ where $M$ is the degree of polynomial $f$ \cite[p.~144]{statistical-learning}. When dealing with polynomials, splines which are continuous up to and including its second order derivatives are called \textit{cubic splines} and they generate nicely looking curves when performing the regression. \cite[p.~143]{statistical-learning}.

\subsubsection{Knots and trigonometric functions}
To model each year with different behavior, in essence all that is needed is to create 9 knots (10 intervals requires 9 splits) and construct the hinge functions.
However, since the functions being used are sines and cosines, there is no guarantee of continuity at the knots (only continuity if the trigonometric function evaluates to zero at the knot).
In grim cases, when multiple discontinuous functions are combined, one can end up with cusps, that is a steep descent/ascend followed by the opposite movement in rapid succession.
If this phenomenon becomes too large, one should consider rejecting the model, this will be discussed in the result section. 
